%\documentclass[french]{beamer}
\documentclass[aspectratio=169]{beamer}
\usepackage[utf8]{inputenc}
\usepackage[T1]{fontenc}
\usepackage{lmodern}
\usepackage{amsmath, amssymb}
\usepackage{babel}
%\usepackage{unicode-math}
\setbeamertemplate{footline}[frame number]

\DeclareFontFamily{U}{wncy}{}
\DeclareFontShape{U}{wncy}{m}{n}{<->wncyr10}{}
\DeclareSymbolFont{mcy}{U}{wncy}{m}{n}
\DeclareMathSymbol{\Sha}{\mathord}{mcy}{"58} 

\definecolor{rouge}{HTML}{DD0000}

%Pour le TITLEPAGE

\title{Appel à projet}
\subtitle{Mardi 7 mars 2023}
\date{}
\author[PAPAZOGLOU - MARTIN]{Nicolas Papazoglou \& Alexis Martin}
\institute[ENSEA]{ENSEA}

\usetheme{ensea}  

\AtBeginSection[]
{
	\begin{frame}{Plan}
		\tableofcontents[currentsection]
	\end{frame}
}

\AtBeginSubsection[]
{
   \begin{frame}
    	\tableofcontents[currentsection,currentsubsection]
   \end{frame}
}

\begin{document}

\begin{frame}
	\titlepage
\end{frame}

\begin{frame}
\begin{minipage}{0.49\textwidth}
	\begin{itemize}
		\item Porteurs du projet : Alexis Martin \& Nicolas Papazoglou
		\item Demande budgétaire : 64 HETD (répartition 50/50)
		\item Budget (DEE) : 300-400€ / maquette
		\item Activité concernée : enseignement d'électrotechnique et d'automatique (AEI, ESE, MSC) dans un premier temps, cours de 2nde année possible dans un second temps.
	\end{itemize}
\end{minipage}
\begin{minipage}{0.49\textwidth}
	\includegraphics[scale=0.6]{inverter.jpeg} 
\end{minipage}

%********************** OBJECTIFS ******************************
\end{frame}
\begin{frame}{Objectif : Réalisation maquette pédagogique}
	Cibles : 
	\begin{itemize}
		\item Enseignements d'électrotechnique et automatique
	\end{itemize}
	Objectifs : 
	\begin{enumerate}
		\item Une maquette fiable et facilement réparable pour les TPs d'électrotechnique et automatique
		\item Projet open-source (disponible sur github),
		\item Compréhension globale possible par les étudiants, application de l'ensemble de leurs cours dans une maquette,
		\item Création modulaire, réutilisable dans d'autres cours/projets, 
		\item Evolution possible à d'autres enseignements (buck/boost, 4Q, brushless, moteurs synchrones, asservissement, etc...).
	\end{enumerate}
\end{frame}

\begin{frame}{Travail à effectuer}
\begin{enumerate}
	\item Cahier des charges (déjà effectué),
	\item Schéma d'architecture (déjà effectué),
	\item Choix des composants (disponibles et actuels pour une maintenibilité la plus longue possible),
	\item Prototypage et tests unitaires électriques,
	\item Réalisation software et tests complets,
	\item Réalisation mécanique (boitier),
	\item Intégration complète,
	\item Documentation (github),
\end{enumerate}

\end{frame}

\begin{frame}{Dead-line et rémunération}
\begin{itemize}
	\item Fonctionnel pour la rentrée de septembre 2023,
	\item Rémunération demandée : 64 HETD,
\end{itemize}

\end{frame}

\end{document}